\documentclass{article}
    \usepackage[margin=0.7in]{geometry}
    \usepackage[parfill]{parskip}
    \usepackage[utf8]{inputenc}
    \usepackage{amsmath,amssymb,amsfonts,amsthm, natbib}

\begin{document}

\begin{center}
    {\Large\textbf{3ACM-DP | Digital Politics}}

This syllabus is tentative and subject to change.

\textit{Winter, 2021}
\end{center}

\begin{itemize}
    \item \textbf{Instructor:} Mickael Temporão
    \item \textbf{E-mail:} \textit{m.temporao@sciencespobordeaux.fr}
    \item \textbf{Office Hours:} Wednesday, 4:00pm to 5:00pm (CET) - B.140
    % \item \textit{Keywords:} social media and politics, digital politics.
\end{itemize}


\section*{Description}

This course aims to introduce students to the influences of big data and digital technologies on governments, parties, and citizens. Human interaction increasingly occurs in spaces mediated by digital systems. The digitalization of societies transforms politics in several ways. For instance, governments use machine learning algorithms to mobilize voters or determine bail. The use of such technologies raises questions of accountability, ethics, fairness, and transparency. This course engages the social, political, and economic implications of big data, artificial intelligence, large-scale experimentation, and increasing automation on society.

\subsection*{Prerequisites}
None.


\section*{Structure}

This course is offered in a fully asynchronous online format. You will use a \textit{Wiki} platform to complete continuous learning activities throughout the course length. The \textit{Wiki} used in this course is the same platform powering \textit{Wikipedia} where thousands of people have been simultaneously collaborating for years!


\subsection*{Assessment}

Students' grade is a function of two main activities participation (40\%) and research (60\%). These are broken down into several milestones to help you incrementally make progress and succeed. Each of the activities will be completed directly on this wiki. You will not have to submit any assignments as the wiki engine will track time and changes for us.

Because you are not held accountable for doing the work via having to see me face-to-face several times a week, it is up to you to keep up with the work and make sure you are actively participating in the weekly activities. You will have to manage your time to successfully achieve the weekly deliverables. It is your responsibility to do the work and keep up. Write reminders to yourself. Schedule alerts in your phone. Have your buddies in the class hold you accountable.

I want you all to be successful but much of that falls on you!

\bibliographystyle{apsr}
\bibliography{/home/mt/references.bib}
\nocite{salganik2019bit}

\end{document}
